% Preamble
\documentclass[12pt, a4paper, twoside]{article}
\usepackage[a4paper, left=0.75in, right=0.75in, top=1in, bottom=1in]{geometry}
\usepackage{lipsum, verbatim, fancyhdr, lastpage, graphicx, hyperref, amsmath}
\usepackage[backend=bibtex]{biblatex}
\graphicspath{{./plots/}}
\addbibresource{ref.bib}
% Top Matter
\hypersetup{
	colorlinks   = true,
	urlcolor     = blue, 
	linkcolor    = blue, 
	citecolor   = red
}
\pagestyle{fancy}
\fancyhead[CO, CE]{CS 725: Foundations of Machine Learning (Autumn 2023) --- Project}
\fancyhead[LO, LE, RO, RE]{}
\fancyfoot[CO, CE]{Page \thepage\ of \pageref{LastPage}}
\fancyfoot[LO, LE, RO, RE]{}
% \bibliographystyle{plain}

\title{\vspace{-0.5in}\textbf{CS 725: Foundations of Machine Learning \\
Midterm Project Report}}
\author{CV Mavericks\footnote{The project has undergone revisions based on the feedback provided on 18/10/2023. Initially, the project focused on \emph{Image-Based Disease Diagnosis using Deep Learning}. The new project concept has received approval from Krishnakant Bhatt (21q050016@iitb.ac.in).}\\
23m2154, 23m2156, 23m2157, 23m2158, 23m2162}
\date{October 31, 2023}

% Main Matter
\begin{document}
\maketitle
\thispagestyle{fancy}

\begin{abstract}
Our project addresses the limitations of current food image datasets in accurately estimating calories. We introduce a food image dataset that includes both food annotations and detailed volume and mass records for each item, along with a calibration reference. Our dataset comprises 2978 images, each accompanied by annotations and essential food metrics. To estimate the calorie content of food in this dataset, we employ a deep learning approach utilizing Faster R-CNN for food detection and a GrabCut algorithm to outline the contours of each food item. With this information, we calculate the volume and subsequently, the calorie content of each food.
\end{abstract}

\section{Problem Statement - Soumen}
\lipsum[1]

\section{Proposed Solution Approach - Soumen}
\lipsum[2]

\section{Code Survey}
\lipsum[3]

\section{Datasets}
\lipsum[4]

\section{Implementation Details - Soumen}
\lipsum[5]

\section{Preliminary Results}
\lipsum[6]

\section{Roadmap - Soumen}
\lipsum[7]\cite{ang_ml}\cite{ddl_book}

\printbibliography
\end{document}


%\begin{align}
%	P [\bar{x} | y=k] &= P[x_1 | y=k] \cdot P[x_2 | y=k] \dots P[x_{10} | y=k] \\
%	&= \prod_{i = 1}^{10} P[x_i | y = k] \\
%	P [y=k | \bar{x}] &= \left(  \prod_{i = 1}^{10} P[x_i | y = k] \right) \cdot P[y=k] \\
%	\hat{y} &= \underset{k = 0, 1, 2}{\text{argmax}} \left(  \prod_{i = 1}^{10} P[x_i | y = k] \right) \cdot P[y=k]
%\end{align}  
%\begin{equation}\label{E:main}
%	\hat{y} = \underset{k = 0, 1, 2}{\text{argmax}} \left[ \sum_{i = 1}^{10} \left( \log (P[x_i | y = k]) \right) + \log (P[y=k]) \right]
%\end{equation}
%
%{
%	\renewcommand{\arraystretch}{2}
%	\begin{table}[p]
%		\begin{center}
%			\begin{tabular}{c c}
%				\hline
%				Output class label $y$ & Prior probability ($P[y = k]$) \\ \hline
%				$y = 0$ & $0.333$ \\ \hline
%				$y = 1$ & $0.333$  \\ \hline
%				$y = 2$& $0.333$ \\ \hline
%			\end{tabular}
%			\caption{Prior probability of the class labels}\label{T:prior}
%		\end{center}
%	\end{table}
%}